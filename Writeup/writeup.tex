\documentclass{article}
\usepackage{graphicx}

\title{Unscrewer}
\author{Leah Liddle}


\begin{document}
\maketitle
\tableofcontents

\section{Introduction}
Magic: The Gathering (MTG) is a deckbuilding Trading Card Game (TCG) produced by Wizards of the Coast (WOTC) in which players use a resource called "mana", generated by "land" cards, to cast "spell" cards. Mana comes in five colours - White, Blue, Black, Red and Green, typically abbreviated to their first letter, with "U" standing in for Blue, and the acronym "WUBRG" referring to all five together - and most spells require one or more specific colours to be cast, meaning that players must chose land cards that maximize the probability that they will be able to generate the colours needed for their spells. While choosing spell cards is a pleasurable creative endeavour, selecting appropriate land cards is simply a matter of picking optimal choices within the limitations of your budget, overall game strategy, and collection. Unscrewer - named after the community term "mana screw", referring to an inability to play your cards due to insufficient mana - is a web app that will automates this process: this software, when presented with a list of spell cards, outputs a list of lands that it has determined will maximize the chance of a player being able to cast their spells. 

\subsection{Overview of Magic the Gathering}
MTG players beging games drawing a "hand" of cards, and draw one additional card per turn. Each turn, they may play one land card. Mana is extracted from a land card via "tapping" it (turning the physical card 90 degrees), and all lands untap at the start of each turn. A tapped land cannot be used to generate mana until it is next untapped. Players may cast any number of spell cards each turn, as long as they can generate enough mana to cast each one. While the specifics of this nomenclature will be important in this writeup, the overall effect of this is that, absent any other cards which accelerate this process, a player each turn has access to N mana, where N is the number of turns so far in the game on which they have managed to play a land. While deckbuilding restrictions and player life totals vary betwen different "Formats", the aforementioned model remains constant. 

Most cards require one or more mana of one or more specific colors along with a quantity of "generic" mana, which many be any color. However, some cards only require generic mana and can be fully paid for with any color, while others only accept specifically colored mana for their entire mana cost. A small subsect of cards require specifically colorless mana. A card's mana cost is reflected in its "pips" - a card with three green pips and a generic 3 pip costs three green mana and 3 mana of any color. 

The land cards Plains, Island, Swamp, Forest and Mountain are called "Basic Lands" and produce W, U, B, R and G respectively. Decks are normally limited to either one or four copies of each card (depending on the Format), but can have any number of basic land cards. A deck requiring only one colour of mana may fill its manabase entirely with Basics, while a deck requring all five may use very few. 

Some lands, called "Utility lands", produce only colorless mana but provide some other useful in-game effect. 

Cards have several "types"."Basic", as described above, is a supertype. "Land" is a type ("Spell", however, is not - Spell cards are split into other types including Instant, Sorcery and Enchantment). Cards may also have several "subtypes" which can be referenced by other cards for synergistic reasons. A card that says "search your library for an X card" is referring to a card whose lists of types includes, but is not limited to, X. In addition to being the names of basic lands, Plains, Island, Swamp and Forest are also land subtypes, and any land with that subtype automatically can be tapped for mana of the respective color. The card "Underground Sea", for example, has Island and Swamp as subtypes, and can tap for Blue or Black mana; it can be selected as a target by any card that targets an "Island" or "Swamp" card, but not by a card that targets a "Basic" card. 

"Strict Betterness" in magic refers to a card which is in all cases better than another. Underground Sea, for example, is strictly better than an Island or a Swamp, as it can tap for either type of mana. However, the similar card, "Watery Grave", which is also an Island/Swamp, but enters tapped (and is thus unusable on its first turn) unless a player pays two life is not strictly better, as if a player has only 1 life, 5 lands and needs to cast a spell that requires 6 mana to save their life, an Island will allow them to cast that spell on that turn, and a Watery Grave will not. However, many cards that are not "Strictly better" than others may still be generally regarded as better than others thanks to widespread player consensus: since paying life for a benefit is generally considered a negligable cost, the ability to access multiple colours of mana means that Watery Grave is generally considered to be better than an Island. I will refer to this as "Agreed Betterness."

Many lands are arranged into "cycles", in which each land is mechanically identical save for offering different colours. "Sacred Foundry", for example, belongs to the same cycle, "Shock Lands," as "Watery Grave", differing only in that it is a Mountain Plains land that provides Red or White mana. Cycles differ from types in that they are not an explicit in-game concept. 

In player vernacular, which I will use in this writeup, an "X Land" is a land belonging to cycle X, while a "X" - a "Basic", an "Island", etc - refers to any card with those types. Although some lands do produce two mana per tap, this is rare, and a land which "taps for two colours" taps for either of those two colours, not both at once.

\subsection{Land Balancing}
"Balancing" refers to the aspect of card design in which a useful card is given drawbacks to prevent it from being strictly better than many other cards. In spells, this is typically enforced via mana cost: a stronger spell costs more mana. 

WOTC generally avoids making lands that are strictly better than basic lands; cards that are, such as Underground Sea, typically reflect an out of fashion design philosophy, and due to age and power are thus rare and expensive. Because of this, most lands that tap for two or more colors of mana are given some sort of drawback. The most popular way to do this is to say that a two-colour land enters tapped, and thus cannot be used on the turn they are played.

While there are several cycles of two or three color lands that simply enter tapped, many can enter untapped if the player meets a specific criteria, or pays a specific resource, when they are played. Battle Lands, for example, enter tapped unless the player controls two or more basic lands. This gives manabases a self-referential character: a battle land, in a deck whose manabase contains many basic lands, is an excellent choice; if the base contains few, it is a poor choice. 

Working around this balancing is the core functionality of Unscrewer: 


\subsection{Overview of Manabase Optimization}
Most strategies to prevent mana screw focus on choosing appropriate spells: a central concept is a deck's "mana curve", referring to the number of spells it has at each mana cost. Choosing appropriate lands is comparatively understudied, as it is a much more numerically complex process - requiring a count of the total number of pips in your deck, and how they are spread across the cards - but also one that is readily addressed by simply available heuristics. Lands can be very easily categorized by both strict and agreed betterness: Shock and Fetch lands have proven their effectivness time and again in competative contexts, contextually OG Dual Lands and Bond Lands are strictly better than any land which enters tapped and taps for two or fewer colors. 





\section{Library and Language Choices}
My choices of langauge and libraries were informed by two main priorities. Due to my short turnaround time, it was important that I use libraries and langauges with substantial community support for web development. Meanwhile, as a usable app, Unscrewer benefits from high performance so as to maximize the number of simulations it can run, but does not need to offer a complex user interface nor store user data, prompting me to favour high-performance tools over complex and scalable ones.

In places where these requirements are at odds, I prioritized the former: my choice of Python as a backend and Javascript as a frontend was driven largely by the popularity of these languages in web design. However, in other decisions, the two requirements informed each other constructively. I chose Flask as a backend web framework as its simplicity made it both easy to learn and reputably faster; contenders like Django are made both slower and more complex due to their abundance of features (FastAPI, potentially lighter and faster than Flask, was discarded due to its smaller userbase and thus relative paucity of learning resources). React, which I chose as my frontend framework, similarly sports a wealth of support resources, and features a Virtual DOM that lowers performance overheads when users make small input changes - a relevant concept here, as users will likely order several simulations with small preferential changes on each one. 

Lacking access to the popular Django ORM, I interacted with my MTG Card database via SQLAlchemy. To fill that database, I opted to create my own script using the Scrython library that would scrape the MTG Database Scryfall, rather than relying on existing projects that compile data for download, such as mtgio API or mtg.json, as Scrython, being a widely used player resource, is kept regularly updated, and my script could be easily re-run to account for new sets or price fluctuations. 

\section{Database Layout}
\subsection{Local vs Online Storage}
Since new cards are regularly released, and card-price - an important deckbuilding consideration - constantly fluctuates, a key feature of my database is updatability. 

Since my backend has direct access to Scryfall via Scrython, I initially considered trivializing this by foregoing a locally stored database, and simply querying player inputs to Scryfall itself. Testing almost immediately showed this to be untenable: although lands could be downloaded in bulk, referencing a player's input cards \(to determine mana value\) required an individual search for each card. Even small decklists required several minutes to parse.

Instead, I opted to create a backend object, the DBManager, called not from the server but from a seperate script, Manage\_Database.py, which, on running, would RESTfully scrape Scryfall and update all tables. Although RESTful considerations made this code slow to run - taking around ten minutes - this is fairly managable on a weekly timeframe, and could be automated in future versions of the software. 



\subsection{Scryfall Shortcomings}
Scryfall does exclude some information which is relevant to Unscrewer. Since the end-goal of my work here would be to have a database script that would update automatically on a timescale, I needed my script to parse the following automatically:
\begin{enumerate}
\item Land Searching Capabilities - usefully, Scryfall lists the mana that each land produces. However, some lands produce no, or only colorless, mana, and instead sacrifice themselves to search your library for a basic land, which is not listed. Helpfully, most of these lands are sorted into cycles, which I marked in the cycles table as "fetch" lands (a standard term in the community). I then gave each land a cached property "true\_produced" which combined the data from scryfall with, if the land belonged to a fetch cycle, any land types mentioned in the text. 
\item Price in GBP - Scryfall lists prices in EUR and USD. Since price fluctuations would prompt most regular updates anyway, I opted to add my own GBP column and calculate it during the scrape using a conversion library.
\item Intermittent price absences - some cards do not have a listed price. I informed my database to list these with a price of minus 1, and WILL FIGURE OUT HOW TO HANDLE THESE.
\item Edge cases - MTG features "Unset" cards, which typically fall comedically outside design norms. "Little Girl", for example, costs half of one white mana \(no other card has a non-integer mana value\). These cards are not popular but not inconceivable to use, as players frequently make comedy decks, and although I experimented with database schema that would incorporate these - for example, storing mana value as a float rather than an int to accommodate Little Girl - I decided that warping my model around a single rarely\-played card was a low-reward development approach. Instead, I equipped my DBManager with an array edge\_cases, and gave it customized handling instructions for each one (Little Girl's mana value, for example, was rounded to one, as this much more closely fits how she would be played)
\item Cycles - Scryfall does not sort lands into cycles. I addressed this by providing each cycle with a regular expression that matched all lands within it, and sorting on download.
\end{enumerate}











\end{document}