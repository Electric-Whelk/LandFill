\chapter{The Interface}
\section{The Input Parser}
In my initial user-research, subjects desired that LandFill support ``round trips''. this means that LandFill can receive a list of cards in the formats output by known online card databases, and output cards in the same format. For this purpose, I created the InputParser class, of which one is created at the start of each session. Test users identified four databases: TappedOut, Moxfield, Archidekt and Deckbox. Since Tappedout and Moxfield both allow for lists to be copied from the deck homepage rather than from the formatted export panel, and Moxfield and Deckbox take inputted cards in the same format, this means that the InputParser has been designed to distinguish between six possible input formats, and return three possible output formats. 

Since Tappedout and Archidekt both support categorization of cards (i.e.,~draw, ramp, win strategy, enemy card removal) with custom lables, the Input Parser stores all cards in a Dict object corresponding to each category, using a default key if none are specified by the user. On completion of the Monte Carlo process, the InputParser a total of six string objects: for each database format (Deckbox and Moxfield being identical), it returns the total decklist including lands (structured into categories if permitted by the database and provided by the user), and the list of provided lands. 

\section{App Layout}
In the intial mockup drafted in \secref{sec:initialusertests}, LandFill consisted of a single homepage. Users in the think-aloud evaluation generally did not understand the difference between the button that confirmed their nonland inputs and the button which commenced the MonteCarlo. On redrafting, I split the design into four pages, outlined below and displayed in Figs. \ref{fig:deckinput}, \ref{fig:preferences}, \ref{fig:progress} and \ref{fig:output}

\subsection{Deck Input (Fig.~\ref{fig:deckinput})}
This page allows the user to list details kept constant throughout the session, i.e.,~cards inputted by the user, and the currency in which the session is to list card prices.The option to overwrite inputted lands with a new manabase was added in response to users in the Think-Aloud evalution who instantly copied a completed deck from a database into LandFill and manually removed lands from the list. 

\myfig[0.35]{frontendpageone.jpg}{Deck Input page.}{deckinput}

\subsection{Preferences (Fig.~\ref{fig:preferences})}
\label{sec:preferences}
The Preferences Page allows users to customize what lands are candidates for inclusion in the deck. The initial mockup had allowed players to exclude lands or whole cycles via a series of tickboxes, but during the Think-Aloud evaluation, users had found this both overwhelming, given the sheer quantity of lands, and unintuitive, as they did not always know what a given cycle did. Moreover, as LandFill can run more quickly if more cards are deemed by the player to be mandatory for inclusion, my design philosophy was to make it is easy as possible for to mark favoured cards, rather than placing the burden of them to list mandatory lands as part of the deck input. This required me to replace the binary input of a land being unchecked (excluded from consideration) or not with a three-way system, by which players could mark cycles as mandatory, possible or forbidden. I addressed all of these issues via a trio of drag-and-drop boxes from the React-Beautiful-DND library~\cite{reactbeautifuldnd}, while also adding a side-panel element which would show a sample card and card list from a draggable cycle object on mouseover. 

Since some lands may behave identically as manabase components but may have additional mechanics that a player may prefer, I created two additional drag and drop boxes, each of which corresponds to a category of identical lands. The frontend does not support dragging between these boxes; rather, they are used to prioritize these lands according to thhe user's preference. Information from this input creates a session-specific amendment to the prioritization hierarchy used in the backend LandPrioritization object. For the preferencse set in Fig.~\ref{fig:preferences}, LandFill will only trial a Bicycle Land if the Typed Dual Land of its colours is either already in the deck or is excluded from consideration (both Bicycle and Typed Dual Lands being cycles of two-colour taplands with basic landtypes).

The panel of preferences set via tickboxes and numerical inputs above the drag-and-drop column are based on comments made by users during the initial mockup testing. Recall from \secref{sec:knapsack} that players did want to factor price considerations into their choices. While it is impractical, as detailed, to allow the player to set a maximum price for the manabase, LandFill gives players the option to exclude all land cards above a certain price. I consider this to be a reasonable, if imperfect, proxy.

Given the complexity of the interface, it is worth touching on the underlying logic. A common theme among test subjects in pre-development evaluation was a lack of a single consistent plan or criterion determining what lands are included, with players generally expressing a range of broad preferences often caveatted by circumstance~---~price, for example, not being a factor if a player already owned an expensive land, even if they were not interested in buying more from the cycle. As another example, a subject who expressed considerable hostility to any land which only ever enters tapped admitted a fondness for the Surveil Land tapland cycle due to its utility effect. I felt it necessary, consequently, to always give players an override option. Individual cards can be added to consideration despite the status of the cycle as a whole via the tickboxes in the side panel, while use of any of the ``Exclude from Consideration'' Filters, which moves multiple drag-and-drop objects to the ``Exclude'' column, does not prevent any individual items so moved from being dragged \textit{out} of that column. Since this means that a cycle or individual card may be categorized either by dragging and dropping or via a filter or override method, the state of any given land is always mapped to one of two ``positiveArrays'', \texttt{positiveArrays.include} or \texttt{positiveArrays.consider}. If it is excluded via any method, it is put in one or more of several ``excludeArrays'', corresponding to the frontend input used to exclude it (i.e.,~whether it was excluded with its whole cycle via the Drag and Drop panel, whether it was excluded for being an off-colour Fetch Land...). When the simulation is run, all lands in any excludeArray are marked as excluded, and all others are marked as either mandatory or possible, depending on which positiveArray they are in. 

Information about the methodology and criteria used by LandFill, including an explanation of the \(P\) metric (see \secref{sec:initialanalysis}), is contained in the labelled panels, and is available on clicking. 


\myfig[0.35]{frontendpagetwo.jpg}{Preferences page, zoomed out to show all content; the rankings panels would normally be offscreen. The side-panel remains stationary as the user scrolls.}{preferences}



\subsection{Progress (Fig.~\ref{fig:progress})}
\label{sec:progress}
 Although initial drafts stayed on the Preferences page while the Optimizer runs, I felt that this was inappropriate for a runtime greater than one minute~---~I did not want the users to think that the program had simply crashed. Moreover, during mockup testing, several users expressed a desire to see the logic being used by the backend, as this would make the results more trustworthy. Although the Preferences Page contains a contextual explanation of the Hill Climbing algorithm, the use of a Progress page allows for this to be more clearly demonstrated: each step in the Hill Climb increment is printed to screen. At the conclusion of the Hill Climb algorithm, the Output page loads automatically. 

\myfig[0.35]{frontendpagetwopfive.jpg}{Progress page}{progress}

As this feature was added late in development, it is done crudely, by polling the backend every two seconds for new updates. 

\subsection{Output (Fig.~\ref{fig:output})}
 The core feature of the Output Page is the textarea which shows the recommended decklist, which can display either the lands added to the deck or the entire decklist for easy export. When the entire decklist is displayed, if the input format included custom card categorizations (i.e.,~from TappedOut or ArchiDekt), these are re-added here, with new lands being added either under a new category, ``Lands'', or under an existing category if one was detected by the InputParser that contained other land cards at input. 

The left hand panel allows users to use the same mousever interface as was employed on the Preferences Page to examine the cards added to their decklist. One subject, during initial user research, had asked to see the lands ranked by performance, as this would inform both their choice of ramp spells and tell them what lands they should replace on the printing of a new cycle.  

\myfig[0.35]{frontendpagefour.jpg}{Output Page.}{output}