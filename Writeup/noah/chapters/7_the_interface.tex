\chapter{The Interface}
\section{The Input Parser}
\label{sec:inputparser}
In my initial user-research, subjects desired that LandFill support ``round trips''; i.e.,~receive a list of cards in the formats outputted by online card databases, and output cards in the same format. To solve this I created the InputParser class. Test users identified four databases: TappedOut, Moxfield, Archidekt and Deckbox. Since Tappedout and Moxfield both allow for lists to be copied from the deck homepage rather than from the formatted export panel, and Moxfield and Deckbox take inputted cards in the same format, the InputParser has been designed to distinguish between six possible input formats, and return three possible output formats. 

Since TappedOut and Archidekt both support categorization of cards (i.e.,~draw, ramp, win strategy, enemy card removal) with custom labels, the Input Parser stores all cards in a Dict object corresponding to each category, using a default key if none are specified by the user. On completion of the Monte Carlo process, the InputParser returns a total of six string objects: for each database format (Deckbox and Moxfield being identical), it returns the total decklist including lands (structured into categories if permitted by the database and provided by the user), and the list of provided lands. The format output can be toggled by the user on the frontend. 

\section{App Layout}
\label{sec:applayout}
LandFill is spread across four pages, outlined below and displayed in Figs. \ref{fig:deckinput}, \ref{fig:preferences}, \ref{fig:progress} and \ref{fig:output}

\subsection{Deck Input (Fig.~\ref{fig:deckinput})}
This page allows the user to list details kept constant throughout the session, i.e.,~cards inputted by the user, currency. The option to overwrite inputted lands with a new manabase was added in response to users in the Think-Aloud evaluation who instantly copied a completed deck from a database into LandFill and manually removed lands from the list. 

\myfig[0.35]{frontendpageone.jpg}{Deck Input page.}{deckinput}

\subsection{Preferences (Fig.~\ref{fig:preferences})}
\label{sec:preferences}
The Preferences Page allows users to customize what lands are candidates for inclusion in the deck. The initial mockup had allowed players to exclude lands or whole cycles via a series of tickboxes, but during the Think-Aloud evaluation, users had found this both overwhelming and unintuitive. As LandFill can run more quickly if more cards are deemed by the player to be mandatory for inclusion, my design philosophy was to make it is easy as possible for to mark favoured cards, rather than placing the burden of them to list mandatory lands as part of the deck input. This required me to replace the binary input of a land being unchecked (excluded from consideration) or not with a three-way system, by which players could mark cycles as mandatory, possible or forbidden. I addressed all of these issues via a trio of drag-and-drop boxes from the React-Beautiful-DND library~\cite{reactbeautifuldnd}, while also adding a side-panel element which would show a sample card image and card list from a draggable cycle object on mouseover. The tickboxes on this side panel allows individual lands to be included or excluded regardless of the cycle's position.

The Drag and Drop columns labelled ``Rank Equivalent Lands'' each represent a category of cycles that have no mechanical difference between them. It allows users to weight otherwise equivalent lands as laid out in \secref{sec:landprioritization}. The frontend does not support dragging between these boxes, but instead records the position of cycles within them. Information from this input creates a session-specific amendment to the prioritization hierarchy used in the backend LandPrioritization object. For the preferences set in Fig.~\ref{fig:preferences}, LandFill will only trial a Bicycle Land if the Typed Dual Land of its colours is either already in the deck or is excluded from consideration (both Bicycle and Typed Dual Lands being cycles of two-colour taplands with basic landtypes).

The ``Exclude from Consideration'' panel above the drag-and-drop column allows cards to be excluded by category (see \secref{sec:landexclusionandpreferences}). Recall also from \secref{sec:knapsack} that users did want to factor price considerations into their choices. While it is impractical, as detailed, to allow the player to set a maximum price for the manabase, LandFill gives players the option to exclude all land cards above a certain price. I considered this to be a reasonable proxy.

Given the complexity of the interface, it is worth touching on the underlying logic. Since, as covered in \secref{sec:landexclusionandpreferences}, test subjects wanted to be able to remove by category rather than just by cycle, underlying logic needed to track not just the eligibility of a land but \textit{how} that eligibility was set, so that if that criterion is unset, the land would return to its original status. Using any of the ``Exclude from Consideration'' filters to move items to the ``Exclude'' column also does not prevent any individual items so moved from being dragged \textit{out} of that column. To track this, any given land is always in one of two ``positiveArrays'', \texttt{positiveArrays.include} or \texttt{positiveArrays.consider}. If it is excluded via any input, it is also put in one or more of several ``excludeArrays'', corresponding to the frontend input used to exclude it (i.e.,~whether it was excluded with its whole cycle via the Drag and Drop panel, whether it was excluded for being an off-colour Fetch Land...). When the simulation is run, all lands in any excludeArray are marked as excluded, and all others are marked as either mandatory or possible, depending on which positiveArray they are in. 

Information about the methodology and criteria used by LandFill, including an explanation of the \(P\) metric (see \secref{sec:initialanalysis}), is contained in the labelled panels, and is available on clicking. 

\myfig[0.35]{frontendpagetwo.jpg}{Preferences page, zoomed out to show all content; the rankings panels would normally be offscreen. The side-panel remains stationary as the user scrolls.}{preferences}



\subsection{Progress (Fig.~\ref{fig:progress})}
\label{sec:progress}
 Although initial drafts stayed on the Preferences page while the Optimizer runs, I felt that this was both inappropriate for a runtime of more than one minute, and a missed chance to establish trust in LandFill's decision-making (see \secref{sec:userresearchtrust}) Moreover, during mockup testing, several users expressed a desire to see the logic being used by the backend, as this would make the results more trustworthy. The Progress page, therefore, prints each Hill Climb increment to screen. At the conclusion of the Hill Climb algorithm, the Output page loads automatically. 

\myfig[0.35]{frontendpagetwopfive.jpg}{Progress page}{progress}

As this feature was added late in development, it is done crudely, by polling the backend every two seconds for new updates. 

\subsection{Output (Fig.~\ref{fig:output})}
 The core feature of the Output Page is the textarea which shows the recommended decklist. This can display either the lands added to the deck or the entire decklist for easy export. When the entire decklist is displayed, if the input format included custom card categorizations (i.e.,~from TappedOut or ArchiDekt), these are re-added here, with new lands being added either under a new category, ``Lands'', or under an existing category if one was detected by the InputParser that contained other land cards at input. 

The left-hand panel allows users to use the same mouseover interface as was employed on the Preferences page to examine the cards added to their decklist, ordered by performance in the final Hill Climb increment, along with their individual scores. 

\myfig[0.35]{frontendpagefour.jpg}{Output Page.}{output}