\chapter{Introduction}
\section{Overview}
Magic: The Gathering (MTG) is a Trading Card Game (TCG) designed by Wizards of the Coast (WOTC). Players take the role of a wizard, whose deck is a library of spells with which they battle one or more similarly equipped opponents. Pursuit of the hobby thus involves mastery of both gameplay and deck construction. While tournament-level participation relies on ``netdecking''~---~copying a decklist with a history of competitive success~---~the hobby is intended to have a significant creative component, with the unranked Commander format gaining popularity in recent years through its focus on unique and personal decks~\cite{CommanderFormatOverview}.

In addition to spell cards, decks also contain ``land'' cards, which generate ``mana'', a resource expended to cast spells. A deck's lands collectively form its ``manabase''. Mana comes in five colours: White, Blue, Black, Red and Green (abbreviated respectively to W, U, B, R and G, or WUBRG collectively). Spells require specific colours, and lands produce one or more colours. A deck's manabase should be chosen to maximize the probability that the player has access to the colours they need. 

LandFill is a web app that automates this aspect of deck building. It models the selection of lands for a deck as a combinatorial optimization problem, where a given manabase yields a measurable performance score for the deck as a whole. This performance score is necessarily approximate: Churchill \textit{et al} have demonstrated that, as it is possible to construct a Turing Machine within an MTG game whose halting is the necessary condition for a player's victory, a deck's winning strategy is indeterminable \cite{churchill2019magic}. Therefore, it is beyond LandFill's capacity to determine what lands support the most decisive plays. Nevertheless, the heuristic that the winning MTG player is typically the one who spends the most mana over the course of the game \cite{KarstenCurve} provides an opening for approximating a performance score via Monte Carlo search. Using a stripped-down MTG simulator, in which the simulated player aims only to spend as much mana as possible each turn, LandFill estimates over iterated games how effective a given manabase is. It then produces successively optimized manabases via a Hill Climbing Algorithm. Users may input a deck's spell cards into this system using a web interface and receive a completed decklist that has been broadly optimized within their set preferences. Initial user testing suggests that users are able to use LandFill to generate manabases of a comparable or greater quality to those they can assemble themselves in less time and with less effort. 

\section{Thesis Layout}
This text documents the development and testing of LandFill. I will begin by outlining relevant MTG rules and design trends, and the difficulties these introduce for optimization, followed by an overview of how LandFill will approach these. I will then outline my initial consultations with MTG players, and the features they require from a manabase optimization app. Dividing LandFill into four components~---~an MTG card database, an MTG game simulator, an optimization algorithm implementer, and a user interface~---~I will then outline its structure, and justify the decisions made during development. I will then summarize a second round of user-testing, and outline the value added by LandFill as well as areas for future development. I will conclude with an evaluation of the strengths and limitations of the app.