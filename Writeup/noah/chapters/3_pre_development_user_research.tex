\chapter{Pre-Development User Research}
\label{chapter:predevelopmentuserresearch}
\section{Overview}
\label{sec:initialusertests}
I began development by holding one-on-one user research sessions with eight MTG players. The sessions were divided into two parts, a Semi-Structured Interview and a Think-Aloud Mockup Test, with an additional Kano Questionnaire circulated to users afterwards. 

\subsection{Semi-Structured Interview}
In addition to specific feedback on possible features, I was interested in understanding more about how players typically create manabases. I therefore chose to lead with a Semi-Structured Interview. This is a data-gathering method in which the interviewer stays flexible on how and in what sequence questions are asked, to allow unexpected themes and topics to emerge~\cite{manuel2004sage}. My questions were as follows:

\begin{itemize}
\item How do you approach selecting lands for a deck, and how does this vary across formats you play?
\item How do you approach acquiring lands for a deck (eg, do you assemble a list of cards to purchase, do you assemble a list of cards you already have~---~and if so, do you have a good knowledge of what lands you own)?
\item What role do existing deckbuilding support apps, such as Moxfield and TappedOut, play in your process?
\item Do you factor the strategy of your deck into land choices in terms of pure mana production (i.e.,~not including utility lands).
\item How do you mulligan? How does this vary across formats that  you play?  
\end{itemize}

\subsection{Think-Aloud Mockup Testing}

In a ``Think-Aloud Evaluation'', users are asked to narrate aloud their thoughts and opinions while attempting to use a system~\cite{wright1991use}. Users are occasionally prompted for input, and may be given solutions to problems if necessary, but are largely expected to use the product unassisted. I provided users with a mockup homepage for LandFill, displayed in Fig.~\ref{fig:mockup}.


\myfig{MockupFrontend.jpg}{Draft front-end used for mockup testing.}{mockup}

\subsection{Kano Questionnaire}
\label{sec:kanoquestionnaire}
Kano Analysis is an approach to user evaluation that examines the emotional response of a prospective user to the presence or absence of a given feature. I developed a Kano Questionnaire after analysis of the Interview and Think-Aloud data, so as to prioritize which features, suggested by individual testers, were reflective of more widespread demand. A generic Kano questionnaire template is displayed in Fig.~\ref{fig:kanoq}. A Kano Analysis Marking Table can then be used to sort features into five categories based on results: ``attractive'' features (presence liked, absence neutral/tolerable), ``must-have'' features (presence expected), ``performance features'' (presence liked, absence disliked, ``indifferent'' features (presence and absence neutral), ``questionable'' features (conflicting thoughts on presence and absence) and ``reverse'' features (presence disliked)). For my Kano Analysis results, see Fig.~\ref{fig:kanoa}

\myfig{KanoSurvey.jpg}{A question in a generic Kano questionnaire}{kanoq}
\mylinefig{KanoAnalysisFull.jpg}{Results from my Kano analysis}{kanoa}

\section{Analysis}
\label{sec:userresearchanalysis}
To analyse the qualitative data from the Think-Aloud Evaluation and the Interview, I implemented a lightweight version of the thematic analysis principles outlined by Braun and Clarke \cite{braun2006using}. I coded interview transcripts to identify recurring themes, which in this context are deckbuilding habits that an app like LandFill should support. As this analysis is intended solely to inform included features, within Braun and Clarke’s framework I favour semantic, inductive and essentialist readings of the data. As a player of Magic: The Gathering myself I have my own approaches to deckbuilding; that extrapolations from my own experience may weight my assessment of this data should be accounted for by the reader. I myself am a Persona B deckbuilder (see \secref{sec:personas}), and favour the Commander format with some experience in Standard and Limited. 

\subsection{Format Support}
This relates to questions 1~---~5 in Fig.~\ref{fig:kanoa}. The formats Limited and Commander were played by all users except one in each case. Other users had experience of Pauper, Modern, Standard and Legacy. However, players in these latter formats commonly expressed doubt that an app such as LandFill would be useful here, as their engagement with those formats rested on netdecking, and often the manabases of decks in those formats were based wholly on relevance to a specific strategy, not on general spell deployment.

A recurring theme, when I asked about deckbuilding approach in Limited, was surprise that the app would be viable in Limited at all. In Limited, decks are built at the site of play from a pool of randomly chosen cards, where there would not be expected to be computer access nor time to input a list of cards. 

LandFill will be restricted to the Commander Format during initial development. Three out of four Kano respondents considered Commander to be a “Performance” feature. The only one who didn’t nonetheless marked that they would expect its presence as minimum. The inclusion of Limited was considered desirable to some degree by three out of four respondents, but only expected as a minimum by one. As Limited requires much stricter pre-setting of what lands can be included – as there are only so many cards available in the random pool – designing for Limited would be a specialized task; so too for Commander, whose rules are detailed in \secref{sec:thecommanderformat}. I consider it better to design a honed app for one Format and potentially expand in the future, with commander being the more popular choice. 

\subsection{Support for 3rd Party Databases}
This relates to questions 6~---~9 in Fig.~\ref{fig:kanoa}. My mockup offered support for TappedOut, Moxfield and Archidekt; one participant also introduced me to Deckbox, of which I had not been previously aware. All Kano respondents said they would ``dislike'' the absence of at least one of these services. LandFill must therefore support them all.

Interestingly, during the Think-Aloud evaluation, while some participants used the dedicated ``export'' features of these apps, but often simply copy + pasted the front page of the decklist from the app to LandFill - this is possible for Moxfield and TappedOut. LandFill therefore must be able to parse decklists from both the export function and the front page of these apps. For implementation of this, see \secref{sec:inputparser}. 

\subsection{Comprehensive Land Exclusion/Preference Options}
\label{sec:landexclusionandpreferences}
This relates to questions 10, 11, 15, 16, 19 and 20 in Fig.~\ref{fig:kanoa}. Several players identified sub-cycles or groups of cycles of lands that they would want to exclude in all cases by category, without having to go through the lands individually. These were:

\begin{itemize}
\item Taplands - lands that always enter tapped under all circumstances. 
\item Off-Colour Fetch Lands - in, for example, a BUG deck, it may be worthwhile to include a UW Fetch Land, as it is able to get UG or UB Shock Lands. One participant found this aesthetically displeasing. 
\end{itemize}

Both have been implemented. Three out of four Kano respondents considered the first ``attractive''; although feedback on the second is mixed, it is simple to implement. Participants generally found it hard to determine lands to exclude without having reminders of what the card text does. This was considered by all Kano respondants to be an ``attractive'', ``must-have'' or ``performance'' feature, and thus should be implemented. Implementation of all three features is discussed in \secref{sec:preferences}.

Two participants suggested offering some ability to ``weight'' lands, to prioritize those that they liked over ones they did not, while keeping options open. Kano data on this was interesting: three respondents felt it to be an attractive feature, but one considered it actively undesirable. This is understandable, given that LandFill's offering is an optimized manabase. It has not been fully implemented. However, given that it was independently suggested by two participants and was otherwise popular with Kano Respondents, it has been used as a framework for dealing with mechanically equivalent lands. This will be covered in \secref{sec:landprioitization}. 

More broadly, most Kano respondents considered it ``attractive'' or a ``must-have'' to be able to specify both lands to \textit{include} and lands to \textit{exclude}. This means that lands must be categorized three ways, with the third being lands for general consideration. This informs the three-column design ultimately used for the frontend, as covered in \secref{sec:preferences}.

\subsection{Deckbuilder Personas}
\label{sec:personas}
Participants broadly designed manabases in one of two ways, embodied in the below personas:

\begin{itemize}
\item Persona A, who possesses a large collection of land cards and, on creation of a new deck, selects lands in the approrpiate colours from this collection.
\item Persona B, who develops a new deck and chooses lands based on abstract preference, and then orders those lands.
\end{itemize}

This prompted me to ask participants which of the following two models of LandFill they would prefer:

\begin{itemize}
\item Model 1~---~LandFill, in addition to taking a decklist of nonland cards, also takes a list of land cards that the player might have found in their collection, and returns the optimum subsection of these.
\item Model 2~---~LandFill takes a decklist of nonland cards and asks the user only to specify what lands they do not want, generating an optimized list from the remaining options.
\end{itemize}

I expected a preference towards Model 1 to be strongly associated with Persona A deckbuilders and vice versa, but to my surprise, a majority of deckbuilders across the personae preferred Model 2. Several Persona A deckbuiders preferred it because they were enthused by the prospect of being recommended lands they had not heard of before. 

While this was a majority consensus, and all Kano respondents said they would be able tolerate the absence of Model 1 functionality, some users did express a desire for Model 1 functionality, making it a viable route for future development of LandFill. However, the initial design covered in this writeup will adhere to Model 2. Since justification of this focus rests on the ability of LandFill to make suggestions the user may not have thought of, that is a meaningful testing criterion for LandFill as a whole. The manner in which that is tested is discussed in \secref{sec:serendipity}. 

\subsection{Visualization of Internal Workings}
This relates to questions 14, 18 and 21 in Fig.~\ref{kanoa}. Several participants felt that they would not automatically trust the output of LandFill, and suggested these features. The participant who suggested adding a ranked list of lands to the output page pointed out that this would allow them to choose what lands to remove if new cycles were printed, suggesting a possible appropriated use of LandFill as not just a generator of manabases, but as a way of learning more about the behaviour of one's manabase. All these features were considered ``attractive'' or ``must-have'' by a majority of kano respondents, and have been implemented on the frontend as will be covered in \secref{sec:applayout}. More broadly, the desire for trust informed both my implementation of live progress updates during optimization (see \secref{sec:progress}), and my choice of objective function (see \secref{sec:propvscum}). 

\subsection{Knapsack Problems and simultaneous optimization}
This relates to questions 12 and 13 in Fig~\ref{kanoa}; regarding question 13, recall from \secref{sec:balancinglands} that some lands are balanced over others by requiring a life point investment, such as Shock Lands compared to Basic Lands. 

These are theoretically easy to determine for any decklist. Manabase cost can be determined from the sum price of each card, data which can be easily included in the Database. While tracking life expenditure would be more difficult, as it would require the Simulator to try and minimize life point expenditure in a situation where there are multiple ways to spend the same amount of mana, it would also be possible to return the average life point penalty incurred over a Monte Carlo search along with the general performance metric. However, these are both examples of the 0---1 Knapsack Problem, itself a NP-hard combinatorial optimization problem, in which a subsection of weighted items must be chosen from a larger set that minimize the total weight (with weight, in this case, representing respective average life damage and card cost)\cite{martello1990knapsack}. Since this would involve simultaneously optimizing multiple areas of deck design, it would represent a sizeable increase in computational complexity for LandFill. They have thus not been implemented in their pure form. However, both of these features were deemed by Kano Respondents to be attractive to some degree, so LandFill implements a lightweight version of each. LandFill considers life loss to be an indicator of strict worseness when that is encoded into its decision making as covered in \secref{sec:landprioritization}. My proxy for overall manabase prices is covered in \secref{sec:preferences}.


\subsection{Mulligans}
\label{sec:mulliganheuristic}
No participant was able to describe any consistent principles on which they based their decision to mulligan. All participants said the decision would be based not just on the castability of spells in the hand, but also the strategy enactable via those cards. It is not, therefore, my priority to provide a means to replicate a given LandFill user's mulligan preferences. For the initial design, the mulligan heuristic will be built into the simulator. 
