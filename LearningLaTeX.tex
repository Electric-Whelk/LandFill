\documentclass{article}
\usepackage[utf8]{inputenc}
\usepackage{amsmath, amssymb, amsthm}
\usepackage{listings}

\title{Learning To Use LaTeX}
\author{Leah}
\date{May 2025}

\begin{document}

\maketitle

\section{Introduction}

So, let's learn how to write code.

\begin{verbatim}
You can write inside the \texttt{verbatim} environment 
and it's printed literally in a block like this. I don't know where the \texttt comes in here, but since I've included, let's highlight that doing this ignores LaTeX commands.
\end{verbatim}

You can also use some syntax you'll see in the raw version of this to write inline code \verb|like this|. Also at around this stage in your writeup you noticed some weird indentations but I think that's just because LaTeX automatically indents the first line of each paragraph. 

Note at the top you've done a lil \verb|\usepackage{listings}|, which is a bit like verbatim but keeps line breaks and whitespaces. Useful for copying in a big block of code like in the python example below:

\begin{lstlisting}
import numpy as np
    
def incmatrix(genl1,genl2):
    m = len(genl1)
    n = len(genl2)
    M = None #to become the incidence matrix
    VT = np.zeros((n*m,1), int)  #dummy variable
    
    #compute the bitwise xor matrix
    M1 = bitxormatrix(genl1)
    M2 = np.triu(bitxormatrix(genl2),1) 

    for i in range(m-1):
        for j in range(i+1, m):
            [r,c] = np.where(M2 == M1[i,j])
            for k in range(len(r)):
                VT[(i)*n + r[k]] = 1;
                VT[(i)*n + c[k]] = 1;
                VT[(j)*n + r[k]] = 1;
                VT[(j)*n + c[k]] = 1;
                
                if M is None:
                    M = np.copy(VT)
                else:
                    M = np.concatenate((M, VT), 1)
                
                VT = np.zeros((n*m,1), int)
    
    return M
\end{lstlisting}

Something else pretty funky is that you can pull code from a file like below, which pulls from the main file in python (i've also tried adding a caption):

\lstinputlisting[language=Python, caption=Main Example]{PythonCode/main.py}

You can also make it pull from some specific lines, as follows:

\lstinputlisting[language=Python, firstline=2, lastline=6]{PythonCode/main.py}

Of course, the tricky thing about this is that it's live; all of this will change as you edit main, and as you shut lines about. We could try this:

\lstinputlisting[language=Python]{PythonCode/main.py}


 


\end{document}